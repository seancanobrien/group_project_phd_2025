\pagebreak %sI will take those out later, for now they will help me to write

\section{Baumslag-Solitar groups and their generalisation}

In this chapter we will explore a certain class of groups, which were initially introduced in \cite{BaSo62}. They have since served as examples and counterexamples of groups with different properties.

\subsection{Definition and first properties}

\begin{definition}
    A \emph{Baumslag-Solitar group $BS(m,n)$} is a group given by the presentation \[BS(m,n) = \langle \: a, t\:|\:t(a^m)t^{-1} = a^n \: \rangle \] where $m,n \in \Z \setminus \{0\}$.
\end{definition}

\begin{remark}
    Note, that for $m = n = 1$, $BS(m,n) = \langle a,t \: | \: tat^{-1} = a \: \rangle \cong \Z \times \Z$. This case seems to be regarded as somehow seperate in the literature.
\end{remark}

The first property of these groups that we will consider, and the one that motivated Baumslag and Solitar in \cite{BaSo62}, is that of being (non-)Hopfian.

\begin{definition}
    A group $G$ is said to be \emph{Hopfian} if every epimorphism from $G$ to itself is injective. In other words, if $G/N \cong G$ implies $N = 1$. Otherwise, we say $G$ is \emph{non-Hopfian}.
\end{definition}
    
Before we proceed, we should mention that examples of Hopfian groups include:
\begin{enumerate}
    \item simple groups;
    \item $(\mathbb{Q},+)$;
    \item finitely generated residually finite groups (by the theorem of Mal'cev), where we say that $G$ is residualy finite if for each $g \neq 1_G$ in $G$, there exists a finite group $F$ and a group homomorphism $\phi: G \to F$ such that $\phi(g) \neq 1_F$.
    \item finitely generated free groups.
\end{enumerate}

Examples (1) - (3) were taken from \cite{CeSi23}. An interested reader can find proofs of (3) and (4) being Hopfian in \cite[~chapters I, IV]{LySch15}. 

\begin{importantexample}\cite[page 514]{BrHa11}
    The group $BS(2,3) = \langle a,t \: | \: t(a^2)t^{-1} = a^3\rangle $ is non-Hopfian. To see that, one considers $\phi: BS(2,3) \to BS(2,3)$ defined on generators as $a \mapsto a^2$, and $t \mapsto t$. Note, that $a = a^3a^{-2} = ta^2t^{-1}a^{-2}$, so $a$ is in the image of $\phi$. Therefore, $\phi$ is onto, but also $[a,tat^{-1}] = atat^{-1}a^{-1}ta^{-1}t^{-1}$ is mapped to identity by $\phi$, thus being an example of a nontrivial element in $ker(\phi)$.
\end{importantexample}

\textcolor{red}{maybe add more properties?}

\subsubsection{Baumslag-Solitar groups as HNN extensions}

In this subsection we will follow the definitions and conventions from \cite[pages 497-498]{BrHa11}.

\begin{definition}
\label{HNN}
    Let $G$ be a group, $\phi: A_1 \to A_2$ an isomorphism between two subgroups $A_1$,$A_2$ of $G$. A \emph{HNN extension of G} associated to that data is the quotient of $G \ast \langle t \rangle$ by the smallest normal subgroup containing $\{a^{-1}t\phi(a)t^{-1} \: | \: a \in A_1 \}$. Thus, we can represent that extension by a relative presentation 
    \[G \ast_\phi = ( G,t \: | \: t^{-1}at = \phi(a), \forall a \in A_1). \]
\end{definition}

\begin{remark}
    If $A$ is an abstract group isomorphic to both $A_1$, $A_2$, then instead of $G \ast _\phi$ we may write $G \ast _A$. We refer to $G \ast _A$ as a 'HNN extension of $G$ over $A$'.
\end{remark}

\begin{example}
    $BS(m,n)$ is a HNN extension of $\Z$. To see this, we consider two subgroups of $\Z$, $m\Z = \{b^m\: | \: b \in \Z\}$ and $n\Z = \{b^n\: | \: b \in \Z\}$ with $m,n \in \Z \setminus \{0\}$. Note, that we are using multiplicative notation for the ease of the later argument, so $b^m$ means ``\emph{add $b$ to itself $m$ times}''. Define $\phi: m\Z \to n\Z$, by $b^m \mapsto b^n$. Then, $\phi$ is an isomorphism, and we can consider $\Z\ast_\phi = (\Z,t \: | \: t^{-1}at = \phi(a), \forall a\in m\Z)$. It is not hard to see though, that because $\Z = \langle a \: | \:\: \rangle$, $\Z\ast_\phi = \langle a,t \: | \: t^{-1}(a^m)t = \phi(a^m), \: \forall a\in \Z \rangle = \langle a,t \: | \: t^{-1}(a^m)t = a^n, \: \forall a\in \Z \rangle $, which is exactly the presentation of $BS(m,n)$.
\end{example}

%One of the reasons that HNN extensions are important is because together with free products with amalgamation they form the building blocks of graphs of groups (in the sense of Bass and Serre).

\subsection{$G$-trees and GBS trees}
%plan for this section - define G-trees and maybe how they connect to graphs of groups; then at some point you can do GBS trees
%you should definitely try to work out how the BS trees abd graphs would look like  -  maybe try to look through trees by Serre

For the preliminary definitions in this subsection we will follow \cite[chapter I, section 2]{Ser80}.

\begin{definition}
    A \emph{graph} $\Gamma$ consists of 
    \begin{itemize}
        \item a set $X = \:vert\:\Gamma$,
        \item a set $Y = \:edge\:\Gamma$,
        \item $Y \to  X \times X$ with $ y \mapsto (o(y), t(y))$, and 
        \item $Y \to Y$ with $y \mapsto \overline{y}$
    \end{itemize}
     which satisfy the following condition: for each $y \in Y$ we have $\overline{\overline{y}} = y$, $\overline{y} \neq y$ and $o(y) = t(\overline{y})$.
\end{definition}

We call elements of $X$ \emph{vertices}, and elements of $Y$ \emph{(oriented) edges}. Given $y \in Y$, the edge $\overline{y}$ is said to be the \emph{inverse edge}. Note, that we can define a morphism of graphs by mapping vertices to vertices, and mapping an edge between two vertices to an edge between their images.

Another notion that we can associate to a graph $\Gamma$ is that of orientation. That is, an \emph{orientation} of $\Gamma$ is a subset $Y_+$ of $Y$ such that $Y$ is a disjoint union of $Y_+$ and $\overline{Y_+}$. We can then define, up to isomorphism, an \emph{oriented graph}, by giving the two sets $X$ and $Y_+$ with a map $Y_+ \to X \times X$. The set of edges $Y$ is the disjoint union we described before.

%Using the concept of the oriented graph, we are able to define the following.

%\begin{definition}
%    Let $G$ be a group, $S$ a subset of $G$. Let $\Gamma(G,S)$ denote the oriented graph which has $G$ as its vertices, $(G \times S) = (edge\: \Gamma)_+$ as its orientation with $o(g,s) = g$ and $t(g,s) = gs$ for each edge $(g,s) \in G \times S$.
%\end{definition}

%Note that the action of $G$ by left multiplication on $\Gamma(G,S)$ preserves orientation and is free on both vertices and edges.

We are almost ready to define trees, which is a class of graphs that will be important later. The only thing we need, is the definition of a circuit.

\begin{definition}
    For an integer $n \ge 1$, $Circ_n$ is an oriented graph with $X = \{0,1, \ldots , n-1 \}$ and edges $Y_+ = \{ y \:| \: (o(y),(t(y)) = (i,i+1), \: i\in \{0,1,\ldots,n-1\} $ where $(n-1, (n-1) + 1)$ is set to $(n-1,0)\}$. A \emph{circuit} (of length $n$) in a graph is any subgraph isomorphic to $Circ_n$.
\end{definition}

\begin{definition} 
    A tree is a connected non-empty graph with no circuits.
\end{definition}

\begin{definition}
    A graph of groups $(G,T)$ consists of a graph $T$, a group $G_p$ for each $p \in vert\:T$, and a group $G_y$ for each $y \in edge\: T$, together with a monomorphism $G_y \to G_{t(y)}$ (denoted $a \mapsto a^y$). In addition, it is required that $G_y = G_{\overline{y}}$.
\end{definition}

If in the above definition $T$ is a tree, then we call $(G,T)$ a tree of groups.

\textcolor{red}{What is the connection between G-trees and graphs of groups? - look at Serre I.5.4}


