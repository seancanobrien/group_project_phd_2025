\subsection{Small categories without loops}

We will first recall the standard construction of the geometric realisation $\Gamma(P)$ of a poset $P$.
Given a poset $P$, a chain $C \subseteq P$ is a totally ordered subposet, i.e.~either $x \leq y$ or  $y \leq x$ for all  $x,y \in C$.
We say $C$ is an $n$--chain if $n=\Abs{C}-1$.
To construct $\Gamma(P)$, we first associate to each $n$--chain in $P$, an $n$--simplex $\Delta_{C}$.
The elements of $C$ correspond to the vertices of $\Delta_{C}$.
Such a $\Delta_C$, and all of its faces, are canonically oriented due to the ordering in $P$.
Using this orientation, we glue simplices corresponding to subchains to a face of the higher dimensional simplex, which corresponds to the superchain.
Specifically, let $C_1 \subset C_2$ be chains, with $C_1$ an $n$--chain.
Let $f$ denote the orientation preserving isometry from (the closed) $\Delta_{C_1}$ to the relevant (closed) $n$--dimensional face of $\Delta_{C_2}$.
We then glue $C_1$ to $C_2$ using  $f$.
Taking $\sim$ to be the transitive closure of the relation defined by all such $f$, for all chains and subchains of  $P$, we then have
\[
	\Gamma(P) \coloneq \left(\;\bigsqcup_{\substack{ C \subseteq P \\ \text{chain}}} \Delta_C \right)/\sim
	.\]

See \cref{fig:example_poset_complex} for an example, where $\Gamma(P)$ is three solid tetrahedrons.
Two tetrahedrons, on the left half of the picture, share a face, and all three tetrahedrons share an edge, which is vertical and centred in the picture.

% filled small circle
\tikzstyle{FSC}=[circle, draw=black!50,fill=black!20,thick, inner sep=0pt,minimum size=1.5mm]
\tikzstyle{light}=[black!22!white]

\begin{figure}
	\centering
	\begin{tikzpicture}
		\tikzstyle{every label}=[font=\footnotesize]
		\tikzstyle{every node}=[font=\footnotesize]
		\node[FSC] (base)		at (0,0)				[label=below:$\emptyset$]			{};
		\node[FSC] (bottom left)	at ($(base) + (-0.8,1) $)   		[label={[label distance=-0.1cm]200:$\{1\}$}]    {};
		\node[FSC] (bottom right)	at ($(bottom left) + (1.6,0)$)  	[label={[label distance=-0.1cm]300:$\{2\}$}]    {};
		\node[FSC] (top left)		at ($(bottom left) + (-1.1,1)$)    	[label=left:{$\{1,4\}$}]          		{};
		\node[FSC] (top middle)    	at ($(base) + (0,2)$)  			[label=right:{$\{1,3\}$}]			{};
		\node[FSC] (top right)    	at ($(bottom right) + (1.1,1)$)  	[label=right:{$\{2,3\}$}]          		{};
		\node[FSC] (top)          	at ($(base) + (0,3)$)    		[label=above:{$\{1,2,3,4\}$}]			{};

		\draw (base) 		to 	node[auto] 		{} 	(bottom left);
		\draw (base) 		to 	node[auto, swap] 	{} 	(bottom right);
		\draw (bottom left)		to 	node[auto] 		{} 	(top left);
		\draw (bottom left)		to 	node[auto] 		{} 	(top middle);
		\draw (top middle)		to 	node[auto]		{}	(top);
		\draw (bottom right)		to 	node[auto, swap] 	{} 	(top right);
		\draw (top left)		to 	node[auto] 		{} 	(top);
		\draw (top right)		to 	node[auto, swap] 	{} 	(top);
	\end{tikzpicture}
	\hspace{1cm}
	\begin{tikzpicture}[baseline=-18pt]
		\node[FSC] (base)          	at (0,0)					{};
		\node[FSC] (bottom left)	at ($(base) + (-0.9,1) $)			{};
		\node[FSC] (bottom right)	at ($(bottom left) + (1.6,0)$)  		{};
		\node[FSC] (top left)		at ($(bottom left) + (-1.1,0.5)$)		{};
		\node[FSC] (top middle)    	at ($(base) + (-0.22,1.15)$)			{};
		\node[FSC] (top right)    	at ($(bottom right) + (1.1,0.5)$)		{};
		\node[FSC] (top)          	at ($(base) + (0,3)$)    			{};

		\draw (base) 			to					(bottom left);
		\draw (base) 			to 	  				(bottom right);
		\draw (bottom left)		to 	 				(top left);
		\draw (bottom left)		to 	 				(top middle);
		\draw (top middle) 		to 					(top);
		\draw (bottom right)	 	to 	  				(top right);
		\draw (top left) 		to 	 				(top);
		\draw (top right) 		to 	  				(top);


		\begin{pgfonlayer}{background}
			\draw[light] (base) 			to 	 				(top middle);
			\draw[light] (base) 			to 	  				(top);
			\draw[light] (base) 			to 	 				(top left);
			\draw[light] (base) 			to 	 				(top right);
			\draw[light] (bottom left) 	to 					(top);
			\draw[light] (bottom right) 	to 	  				(top);
		\end{pgfonlayer}
	\end{tikzpicture}
	\caption{The diagram of a poset where $\leq$ is $\subseteq$ (left) and a picture of the corresponding complex $\Gamma(P)$ (right) with the original poset highlighted in black.}
	\label{fig:example_poset_complex}
\end{figure}

Posets, and the construction $\Gamma(P)$, give a combinatorial description of certain simplicial complexes.
However, not all complexes can be realised by this construction.
For instance, consider $S^1$ realised as two edges connected at their ends.
This limitation is linked to the following observation.

All the information of a poset $P$ is encoded exactly by the following category $\catc$.
The objects of  $\catc$ are the elements of  $P$, and there is a morphism  $x \to y$ in  $\catc$ exactly when  $x \leq y$ in $P$.
The category $\catc$ is \emph{thin}, meaning that for all  $x,y \in \ob(\catc)$, there is at most one morphism  $x \to y$.
Bearing this in mind, with the aim of being able to construct $S^1$ in a combinatorial way, we give the following definition.

\begin{definition}
	A small category without loops (abbreviated scwol), is a small category $\catx$ such that for all composable morphisms $x_1 \to x_2 \to \cdots \to x_n$, if $x_1=x_n$, then $x_1=x_i$ for all $i$, and each morphism is $\id_{x_1}$.
	\label{def:scwol}
\end{definition}

Note that this definition means that no two distinct objects are isomorphic.
It also means that the only morphism $x \to x$ is $\id_x$.
Conversely, if a small category satisfies both of these properties, then it is a scwol.

With geometry in mind, we call the objects of a scwol $\catx$ \emph{vertices} and denote the set of vertices by $V(\catx)$.
We call the morphisms of  $\catx$ \emph{edges} and denote the set of edges by $E(\catx)$.
Given some $a \in E(\catx)$, we denote the initial (source) and terminal (target) vertices by $i(a)$ and $t(a)$ respectively.
If two edges $b$ and $a$ satisfy $t(b)=i(a)$, then we say that $a$ and $b$ are composable and denote their composition $ab$, which is necessarily another edge in $E(\catx)$.
We have that $i(ab) = i(b)$ and $t(ab) = t(a)$.

We now work to construct a geometric realisation $\Delta(\catx)$ of a scwol $\catx$, such that if $\catx$ is thin, then $\Delta(\catx)$ matches $\Gamma(\catx)$, considering $\catx$ as a poset.

Let $E^{(n)}(\catx)$ denote the set of all $n$--tuples of composable edges in $\catx$,
i.e.~if $(a_1, a_2, \ldots, a_n) \in E^{(n)}(\catx)$, then
\[
	\begin{tikzcd}
		x_1 \ar[r, "a_1"] & x_2 \ar[r, "a_2"] & {} \ar[r, phantom, "\cdots"] & {} \ar[r, "a_n"] & x_{n+1}
	\end{tikzcd}
\]
is a composable set of edges in $\catx$.

We define the following \emph{face maps}, $\partial_i \colon E^{(n)}(\catx) \to E^{(n-1)}(\catx)$.
\begin{eqnarray*}
	\partial_0(a_1,\ldots,a_n) =& (a_2,\ldots,a_n) \\
	\partial_i(a_1,\ldots,a_n) =& (a_1, \ldots, a_ia_{i+1}, \ldots, a_n) \\
\end{eqnarray*}

\begin{definition}
	The nerve of a category is the set of
\end{definition}
yo





