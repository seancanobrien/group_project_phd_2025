\subsection{Small categories without loops}

We will first recall the standard construction of the geometric realisation $\abs{P}$ of a poset $P$.
Given a poset $P$, a chain $C \subseteq P$ is a totally ordered subposet, i.e.~either $x \leq y$ or  $y \leq x$ for all  $x,y \in C$.
We say $C$ is an $n$--chain if $n=\abs{C}-1$.
We denote the standard $k$--simplex as $\Delta^k$, where
\[
	\Delta^k \coloneq \Set{(x_0,\ldots, x_k) \in \R^{k+1} \given x_i \geq 0 \; \forall i \text{ and } x_0 + \cdots + x_k =1}
	.\]

To construct $\abs{P}$, we first associate to each $n$--chain in $P$, an $n$--simplex $\Delta_{C}$.
The elements of $C$ correspond to the vertices of $\Delta_{C}$.
Such a $\Delta_C$, and all of its faces, are canonically oriented due to the ordering in $P$.
Using this orientation, we glue simplices corresponding to subchains to a face of the higher dimensional simplex, which corresponds to the superchain.
Specifically, let $C_1 \subset C_2$ be chains, with $C_1$ an $n$--chain.
Let $f$ denote the orientation preserving isometry from (the closed) $\Delta_{C_1}$ to the relevant (closed) $n$--dimensional face of $\Delta_{C_2}$.
We then glue $\Delta_{C_1}$ to $\Delta_{C_2}$ using  $f$.
Taking $\sim$ to be the transitive closure of the relation defined by all such $f$, for all chains and subchains of  $P$, we then have
\[
	\abs{P} \coloneq \left(\;\bigsqcup_{\substack{ C \subseteq P \\ \text{chain}}} \Delta_C \right)/\sim
	.\]

See \cref{fig:example_poset_complex} for an example, where $\abs{P}$ is three solid tetrahedrons.
Two tetrahedrons, on the left half of the picture, share a face, and all three tetrahedrons share an edge, which is vertical and centred in the picture.

% filled small circle
\tikzstyle{FSC}=[circle, draw=black!50,fill=black!20,thick, inner sep=0pt,minimum size=1.5mm]
\tikzstyle{light}=[black!22!white]

\begin{figure}
	\centering
	\begin{tikzpicture}
		\tikzstyle{every label}=[font=\footnotesize]
		\tikzstyle{every node}=[font=\footnotesize]
		\node[FSC] (base)		at (0,0)				[label=below:$\emptyset$]			{};
		\node[FSC] (bottom left)	at ($(base) + (-0.8,1) $)   		[label={[label distance=-0.1cm]200:$\{1\}$}]    {};
		\node[FSC] (bottom right)	at ($(bottom left) + (1.6,0)$)  	[label={[label distance=-0.1cm]300:$\{2\}$}]    {};
		\node[FSC] (top left)		at ($(bottom left) + (-1.1,1)$)    	[label=left:{$\{1,4\}$}]          		{};
		\node[FSC] (top middle)    	at ($(base) + (0,2)$)  			[label=right:{$\{1,3\}$}]			{};
		\node[FSC] (top right)    	at ($(bottom right) + (1.1,1)$)  	[label=right:{$\{2,3\}$}]          		{};
		\node[FSC] (top)          	at ($(base) + (0,3)$)    		[label=above:{$\{1,2,3,4\}$}]			{};

		\draw (base) 		to 	node[auto] 		{} 	(bottom left);
		\draw (base) 		to 	node[auto, swap] 	{} 	(bottom right);
		\draw (bottom left)		to 	node[auto] 		{} 	(top left);
		\draw (bottom left)		to 	node[auto] 		{} 	(top middle);
		\draw (top middle)		to 	node[auto]		{}	(top);
		\draw (bottom right)		to 	node[auto, swap] 	{} 	(top right);
		\draw (top left)		to 	node[auto] 		{} 	(top);
		\draw (top right)		to 	node[auto, swap] 	{} 	(top);
	\end{tikzpicture}
	\hspace{1cm}
	\begin{tikzpicture}[baseline=-18pt]
		\node[FSC] (base)          	at (0,0)					{};
		\node[FSC] (bottom left)	at ($(base) + (-0.9,1) $)			{};
		\node[FSC] (bottom right)	at ($(bottom left) + (1.6,0)$)  		{};
		\node[FSC] (top left)		at ($(bottom left) + (-1.1,0.5)$)		{};
		\node[FSC] (top middle)    	at ($(base) + (-0.22,1.15)$)			{};
		\node[FSC] (top right)    	at ($(bottom right) + (1.1,0.5)$)		{};
		\node[FSC] (top)          	at ($(base) + (0,3)$)    			{};

		\draw (base) 			to					(bottom left);
		\draw (base) 			to 	  				(bottom right);
		\draw (bottom left)		to 	 				(top left);
		\draw (bottom left)		to 	 				(top middle);
		\draw (top middle) 		to 					(top);
		\draw (bottom right)	 	to 	  				(top right);
		\draw (top left) 		to 	 				(top);
		\draw (top right) 		to 	  				(top);


		\begin{pgfonlayer}{background}
			\draw[light] (base) 			to 	 				(top middle);
			\draw[light] (base) 			to 	  				(top);
			\draw[light] (base) 			to 	 				(top left);
			\draw[light] (base) 			to 	 				(top right);
			\draw[light] (bottom left) 	to 					(top);
			\draw[light] (bottom right) 	to 	  				(top);
		\end{pgfonlayer}
	\end{tikzpicture}
	\caption{The diagram of a poset where $\leq$ is $\subseteq$ (left) and a picture of the corresponding complex $\abs{P}$ (right) with the original poset highlighted in black.}
	\label{fig:example_poset_complex}
\end{figure}

Posets, and the construction $\abs{P}$, give a combinatorial description of certain simplicial complexes.
However, not all complexes can be realised by this construction.
For example, consider $S^1$ realised as two edges connected at their ends.
This limitation is linked to the following observation.
\todo[color=olive]{Maybe you could make "Observation" an environment to make the first $\frac{2}{3}$ of following paragraph stand out more}

All the information of a poset $P$ is encoded exactly by the following category $\calc$:
The objects of  $\calc$ are the elements of  $P$, and there is a morphism  $x \to y$ in  $\calc$ exactly when  $x \leq y$ in $P$.
The category $\calc$ is \emph{thin}, meaning that for all  $x,y \in \ob(\calc)$, there is at most one morphism  $x \to y$.
Bearing this in mind, with the aim of being able to construct $S^1$ in a combinatorial way, we give the following definition.

\begin{definition}
	A small category without loops (abbreviated scwol), is a small category $\calx$ such that for all composable morphisms $x_1 \to x_2 \to \cdots \to x_n$, if $x_1=x_n$, then $x_1=x_i$ for all $i$, and each morphism is $\id_{x_1}$.
	\label{def:scwol}
\end{definition}

\todo[color=olive]{Is it assumed that readers know what "small" means in the context of categories?}

Note that this definition means that no two distinct objects are isomorphic.
It also means that the only morphism $x \to x$ is $\id_x$.
Conversely, if a small category satisfies both of these properties, then it is a scwol.

With geometry in mind, we call the objects of a scwol $\calx$ \emph{vertices} and denote the set of vertices by $V(\calx)$.
We call the non--identity morphisms of  $\calx$ \emph{edges} and denote the set of edges by $E(\calx)$.
Given some $a \in E(\calx)$, we denote the initial (source) and terminal (target) vertices by $i(a)$ and $t(a)$ respectively.
If two edges $b$ and $a$ satisfy $t(b)=i(a)$, then we say that $a$ and $b$ are composable and denote their composition $ab$, which is necessarily another edge in $E(\calx)$.
We have that $i(ab) = i(b)$ and $t(ab) = t(a)$.
We now work to construct a geometric realisation $\abs{\calx}$ of a scwol $\calx$, such that if $\calx$ is thin, then $\abs{\calx}$ matches the geometric realisation when considering $\calx$ as a poset.

Let $E^{(n)}(\calx)$ denote the set of all $n$--tuples of composable edges in $\calx$,
i.e.~if $(a_1, a_2, \ldots, a_n) \in E^{(n)}(\calx)$, then
\[
	\begin{tikzcd}
		x_1 \ar[r, "a_1"] & x_2 \ar[r, "a_2"] & {} \ar[r, phantom, "\cdots"] & {} \ar[r, "a_n"] & x_{n+1}
	\end{tikzcd}
\]
is a composable set of edges in $\calx$.
Since all the $a_i$ are in $E(\calx)$, none are identity morphisms.
By convention, $E^{(0)}(\calx) = V(\calx)$.
We define the following maps, $\partial_i \colon E^{(n)}(\calx) \to E^{(n-1)}(\calx)$ on composable tuples, where
\begin{align*}
	\partial_0(a_1,\ldots,a_n) & = (a_2,\ldots,a_n)                                          \\
	\partial_i(a_1,\ldots,a_n) & = (a_1, \ldots, a_ia_{i+1}, \ldots, a_n) \quad 1 \leq i < k \\
	\partial_n(a_1,\ldots,a_n) & = (a_1, \ldots, a_{n-1})
	.\end{align*}

\todo[color=olive]{It seems like $\partial_i$ has too many coordinates as outputs -  shouldn't it be $(a_1, \ldots, a_{i-1},a_{i+1},\ldots,a_n$? Also, shouldn't $i$ be in range $1 \le i <n$?}
We define $\partial_0(a)=i(a)$, and $\partial_1(a)=t(a)$.
We also define maps, $d_i \colon \Delta^{k-1} \to \Delta^k$ on simplices, where
\[
	d_i(t_0, \ldots, t_{k-1}) = (t_0, \ldots, t_{i-1}, 0, t_i, \ldots, t_{k-1}) \quad 0 \leq i < k
	.\]

\begin{definition}
	Given a scwol $\calx$, the geometric realisation, $\abs{\calx}$ is a simplicial complex defined as
	\[
		\abs{\calx} \coloneq \bigsqcup_k \left( \Delta^k \times E^{(k)}(\calx) \right) / \sim
		.\]
	Where $\sim$ is the transitive closure of the relations
	\[
		(d_i(x),t) \sim (x, \partial_i(t))
	\]
	across all tuples $t \in E^{(k)}(\calx)$.
	\label{def:geometric_realisation_of_scwol}
\end{definition}
In this definition, we had to take more care than with the geometric realisation of a poset.
This is because, in a poset, a composable tuple is completely determined by its vertices (the chain), so the combinatorial face maps always corresponded to taking subsets of those vertices.
But in a scwol, we need to account for potentially multiple edges between two vertices, so we must enumerate simplices using edges, rather than vertices.

For those interested, the above construction of $\abs{\calx}$ corresponds to the standard geometric realisation of the nerve (which corresponds to the union of all of our $E^{(k)}(\calx)$) of $\calx$.
See \cite{goerss_simplicial_2009}.

Given a scwol $\calx$, the dimension of  $\calx $ is the dimension of $\abs{\calx}$, this is the largest  $k$ such that  $E^{(k)}(\calx) \neq \emptyset$.

\begin{example}
	The following scwol has geometric realisation $S^1$, constructed by identifying the ends of two 1--simplices.
	\begin{center}
		\begin{tikzpicture}
			\tikzstyle{every label}=[font=\footnotesize]
			\tikzstyle{every node}=[font=\footnotesize]
			\node[FSC] (base)	at (0,0)	{};
			\node[FSC] (top)	at (0,1)	{};
			\draw[arrow_me=stealth, bend left=90] (base) to (top);
			\draw[arrow_me=stealth, bend right=90] (base) to (top);
		\end{tikzpicture}
	\end{center}
\end{example}

\begin{example}
	Any polygon can be realised as the geometric realisation of a scwol $\calx$ in the following way:
	Consider the polygon as a polygonal complex.
	Make a vertex for at the centre of each polygon in the complex.
	\todo[color=olive]{What is "for" doing in this sentence?}
	There is an edge from centres of polygons to centres of faces of that polygon.
	The below picture shows the scwol from this construction applied to a square.
	\begin{center}
		\begin{tikzpicture}
			\def\sqsize{2}
			\tikzstyle{every label}=[font=\footnotesize]
			\tikzstyle{every node}=[font=\footnotesize]
			\node[FSC] (bl)	at (0,0)	{};
			\node[FSC] (br)	at ($(bl)+(\sqsize,0)$)	{};
			\node[FSC] (tl)	at ($(bl)+(0,\sqsize)$)	{};
			\node[FSC] (tr)	at ($(tl)+(\sqsize,0)$)	{};

			\node[FSC] (mblbr) at ($(bl)!0.5!(br)$) {};
			\node[FSC] (mbltl) at ($(bl)!0.5!(tl)$) {};
			\node[FSC] (mbrtr) at ($(br)!0.5!(tr)$) {};
			\node[FSC] (mtltr) at ($(tl)!0.5!(tr)$) {};

			\node[FSC] (center) at ($(bl)!0.5!(tr)$) {};

			\draw[arrow_me=stealth] (center) to (bl);
			\draw[arrow_me=stealth] (center) to (br);
			\draw[arrow_me=stealth] (center) to (tl);
			\draw[arrow_me=stealth] (center) to (tr);

			\draw[arrow_me=stealth] (center) to (mblbr);
			\draw[arrow_me=stealth] (center) to (mbltl);
			\draw[arrow_me=stealth] (center) to (mtltr);
			\draw[arrow_me=stealth] (center) to (mbrtr);

			\draw[arrow_me=stealth] (mblbr) to (bl);
			\draw[arrow_me=stealth] (mblbr) to (br);
			\draw[arrow_me=stealth] (mbltl) to (bl);
			\draw[arrow_me=stealth] (mbltl) to (tl);
			\draw[arrow_me=stealth] (mbrtr) to (br);
			\draw[arrow_me=stealth] (mbrtr) to (tr);
			\draw[arrow_me=stealth] (mtltr) to (tl);
			\draw[arrow_me=stealth] (mtltr) to (tr);
		\end{tikzpicture}
	\end{center}
	\label{eg:scwol_for_square}
\end{example}

\begin{example}
	Following the same procedure as the previous example, the scwol associated to the complete graph on three vertices is given below.
	\begin{center}
		\begin{tikzpicture}
			\def\trisize{1.5}
			\tikzstyle{every label}=[font=\footnotesize]
			\tikzstyle{every node}=[font=\footnotesize]

			\node[FSC] (bl)	at (0,0)	{};
			\node[FSC] (br)	at ($(bl)+(\trisize,0)$)	{};
			\node[FSC] (top) at ($(bl)+(0.5*\trisize,0.833*\trisize)$)	{};

			\node[FSC] (mblbr) at ($(bl)!0.5!(br)$) {};
			\node[FSC] (mblt) at ($(bl)!0.5!(top)$) {};
			\node[FSC] (mbrt) at ($(br)!0.5!(top)$) {};

			\draw[arrow_me=stealth] (mblbr) to (bl);
			\draw[arrow_me=stealth] (mblbr) to (br);
			\draw[arrow_me=stealth] (mblt) to (bl);
			\draw[arrow_me=stealth] (mblt) to (top);
			\draw[arrow_me=stealth] (mbrt) to (br);
			\draw[arrow_me=stealth] (mbrt) to (top);
		\end{tikzpicture}
	\end{center}
	\label{eg:K3_scwol}
\end{example}

\todo[color=olive]{A moral question: do we actually need scwols for the above example, i.e.
the triangle? }

\begin{definition}
	Morphisms of scwols are exactly functors $F \colon \calx \to \caly$ where the source and target categories are scwols.
	We say that a morphism of scwols $F \colon \calx \to \caly$ is non-degenerate if $F$ maps $E(\calx)$ to  $E(\caly)$, and for all $v \in V(\calx)$, the restriction of  $F$ to $\Set{a \in E(\calx) \given i(a) = v}$ is a bijection on that set of edges.
\end{definition}


The requirement that $E(\calx)$ maps to $E(\caly)$ means that non-identity morphisms must be mapped to non-identity morphisms.
A morphism of scwols $F$ induces a map on the geometric realisations, we denote this $\abs{F}$.
This acts on each simplex $\Delta$ by the linear map determined by the image of the vertices of $\Delta$.
Non-degenerate maps are important because they do not reduce the length of compositions of non-identity maps, thus they preserve dimension.
The importance of the second condition is that if a group acts by non-degenerate morphisms, then the stabiliser of $i(a)$ is contained in the stabiliser of $a$, which is contained in the stabiliser of $t(a)$ by virtue of $F$  being a functor.
In this way, this condition guarantees that stabilisers respect the category structure.
The importance of this will become apparent when we define complexes of groups.

(Non-degenerate) automorphisms of a scwols are (non-degenerate) invertible morphisms $F \colon \calx \to \calx$.

\begin{definition}
	An action of a group $G$ on a scwol $\calx$ is a homomorphism from $G$ to the group of non-degenerate automorphisms of $\calx$ such that the following two conditions are met.
	\begin{enumerate}
		\item For all $a \in E(\calx)$ and $g \in G$, we have  $g \cdot i(a) \neq t(a)$.
		\item For all $a \in E(\calx)$ and $g \in G$, if  $g\cdot i(a)=i(a)$, then  $g\cdot a = a$.
	\end{enumerate}
	\label{def:action_of_groups_on_scwols}
\end{definition}
\todo{compare this def to that of actions without inversions in graphs of groups}

If $\calx$ is finite dimensional, then the first condition is guaranteed.
For example, in the following scwol, if $\sigma$ was mapped to  $\tau$, then $\tau$ is mapped to $\mu$ and there is nowhere for $a$ to map to.

\begin{center}
	\begin{tikzpicture}
		\def\size{0.9}
		\tikzstyle{every label}=[font=\footnotesize]
		\tikzstyle{every node}=[font=\footnotesize]

		\node[FSC, label=below:{$\sigma$}] (l) at (0,0)		{};
		\node[FSC, label=below:{$\tau$}] (m) at ($(l)+(\size,0)$)	{};
		\node[FSC, label=below:{$\mu$}] (r) at ($(m)+(\size,0)$)	{};

		\draw[arrow_me=stealth] (l) to node[below] {$b$} (m);
		\draw[arrow_me=stealth] (m) to node[below] {$a$} (r);
		\draw[arrow_me=stealth, bend left=90] (l) to node[above] {$ab$} (r);
	\end{tikzpicture}
\end{center}

However, if we consider the action of the group $\Z$ on the poset $\Z$, we see that  the first condition is not vacuous.

For each $g \in G$, denote the induced action on  $\abs{\calx}$ as $\abs{g}$.
The second condition means that if for some $\sigma \in V(\calx)$, we have $g \cdot \sigma = \sigma$, then $\abs{g}$ fixes pointwise any $k$-simplex corresponding to a composable tuple $(a_1,  \ldots, a_k)$ with $i(a_1)=\sigma$.
This in particular implies that if $\abs{g}$ fixes a simplex setwise, then it also fixes it pointwise.
This is important because we want to be able to quotient by this group action and for the quotient map \todo[color=olive]{Is $\calx$ written in the right font? } to be simplicial on $\abs{X}$.
Note that these restrictions mean that many group actions that geometrically look like rotations about a point, or reflections, are not possible.
For instance, there is no non-trivial group action on the scwol given in \cref{eg:scwol_for_square}.
However, there is an action of the cyclic group of order 3 on the scwol given in \cref{eg:K3_scwol}.

Given a scwol $\calx$, a group $G$, and an action as defined in \cref{def:action_of_groups_on_scwols}, we can define the quotient scwol $G \backslash \calx$ in the following way.
The vertices of $G\backslash \calx$ are $G\backslash V(\calx)$, similarly the edges of $G \backslash \calx$ are $G\backslash E(\calx)$.
Let $p \colon \calx \to G \backslash \calx$ be the quotient map.
We make $G \backslash \calx$ in to a scwol by defining $i(p(a)) \coloneq p(i(a))$, $t(p(a)) \coloneq p(t(a))$, and where defined, $p(b)p(a)\coloneq p(ba)$.
Condition (1) of \cref{def:action_of_groups_on_scwols} ensures that a quotient  of a scwol is also a scwol.
Condition (2) ensures that $p$ is a non-degenerate morphism of scwols.


