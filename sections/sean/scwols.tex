\subsection{Small categories without loops}

We will first recall the standard construction of the geometric realisation $\Abs{P}$ of a poset $P$.
Given a poset $P$, a chain $C \subseteq P$ is a totally ordered subposet, i.e.~either $x \leq y$ or  $y \leq x$ for all  $x,y \in C$.
We say $C$ is an $n$--chain if $n=\Abs{C}-1$.
We denote the standard $k$--simplex as $\Delta^k$, where
\[
	\Delta^k \coloneq \Set{(x_0,\ldots, x_k) \in \R^{k+1} \given x_i \geq 0 \; \forall i \text{ and } x_0 + \cdots + x_k =1}
	.\]

To construct $\Abs{P}$, we first associate to each $n$--chain in $P$, an $n$--simplex $\Delta_{C}$.
The elements of $C$ correspond to the vertices of $\Delta_{C}$.
Such a $\Delta_C$, and all of its faces, are canonically oriented due to the ordering in $P$.
Using this orientation, we glue simplices corresponding to subchains to a face of the higher dimensional simplex, which corresponds to the superchain.
Specifically, let $C_1 \subset C_2$ be chains, with $C_1$ an $n$--chain.
Let $f$ denote the orientation preserving isometry from (the closed) $\Delta_{C_1}$ to the relevant (closed) $n$--dimensional face of $\Delta_{C_2}$.
We then glue $\Delta_{C_1}$ to $\Delta_{C_2}$ using  $f$.
Taking $\sim$ to be the transitive closure of the relation defined by all such $f$, for all chains and subchains of  $P$, we then have
\[
	\Abs{P} \coloneq \left(\;\bigsqcup_{\substack{ C \subseteq P \\ \text{chain}}} \Delta_C \right)/\sim
	.\]

See \cref{fig:example_poset_complex} for an example, where $\Abs{P}$ is three solid tetrahedrons.
Two tetrahedrons, on the left half of the picture, share a face, and all three tetrahedrons share an edge, which is vertical and centred in the picture.

% filled small circle
\tikzstyle{FSC}=[circle, draw=black!50,fill=black!20,thick, inner sep=0pt,minimum size=1.5mm]
\tikzstyle{light}=[black!22!white]

\begin{figure}
	\centering
	\begin{tikzpicture}
		\tikzstyle{every label}=[font=\footnotesize]
		\tikzstyle{every node}=[font=\footnotesize]
		\node[FSC] (base)		at (0,0)				[label=below:$\emptyset$]			{};
		\node[FSC] (bottom left)	at ($(base) + (-0.8,1) $)   		[label={[label distance=-0.1cm]200:$\{1\}$}]    {};
		\node[FSC] (bottom right)	at ($(bottom left) + (1.6,0)$)  	[label={[label distance=-0.1cm]300:$\{2\}$}]    {};
		\node[FSC] (top left)		at ($(bottom left) + (-1.1,1)$)    	[label=left:{$\{1,4\}$}]          		{};
		\node[FSC] (top middle)    	at ($(base) + (0,2)$)  			[label=right:{$\{1,3\}$}]			{};
		\node[FSC] (top right)    	at ($(bottom right) + (1.1,1)$)  	[label=right:{$\{2,3\}$}]          		{};
		\node[FSC] (top)          	at ($(base) + (0,3)$)    		[label=above:{$\{1,2,3,4\}$}]			{};

		\draw (base) 		to 	node[auto] 		{} 	(bottom left);
		\draw (base) 		to 	node[auto, swap] 	{} 	(bottom right);
		\draw (bottom left)		to 	node[auto] 		{} 	(top left);
		\draw (bottom left)		to 	node[auto] 		{} 	(top middle);
		\draw (top middle)		to 	node[auto]		{}	(top);
		\draw (bottom right)		to 	node[auto, swap] 	{} 	(top right);
		\draw (top left)		to 	node[auto] 		{} 	(top);
		\draw (top right)		to 	node[auto, swap] 	{} 	(top);
	\end{tikzpicture}
	\hspace{1cm}
	\begin{tikzpicture}[baseline=-18pt]
		\node[FSC] (base)          	at (0,0)					{};
		\node[FSC] (bottom left)	at ($(base) + (-0.9,1) $)			{};
		\node[FSC] (bottom right)	at ($(bottom left) + (1.6,0)$)  		{};
		\node[FSC] (top left)		at ($(bottom left) + (-1.1,0.5)$)		{};
		\node[FSC] (top middle)    	at ($(base) + (-0.22,1.15)$)			{};
		\node[FSC] (top right)    	at ($(bottom right) + (1.1,0.5)$)		{};
		\node[FSC] (top)          	at ($(base) + (0,3)$)    			{};

		\draw (base) 			to					(bottom left);
		\draw (base) 			to 	  				(bottom right);
		\draw (bottom left)		to 	 				(top left);
		\draw (bottom left)		to 	 				(top middle);
		\draw (top middle) 		to 					(top);
		\draw (bottom right)	 	to 	  				(top right);
		\draw (top left) 		to 	 				(top);
		\draw (top right) 		to 	  				(top);


		\begin{pgfonlayer}{background}
			\draw[light] (base) 			to 	 				(top middle);
			\draw[light] (base) 			to 	  				(top);
			\draw[light] (base) 			to 	 				(top left);
			\draw[light] (base) 			to 	 				(top right);
			\draw[light] (bottom left) 	to 					(top);
			\draw[light] (bottom right) 	to 	  				(top);
		\end{pgfonlayer}
	\end{tikzpicture}
	\caption{The diagram of a poset where $\leq$ is $\subseteq$ (left) and a picture of the corresponding complex $\Abs{P}$ (right) with the original poset highlighted in black.}
	\label{fig:example_poset_complex}
\end{figure}

Posets, and the construction $\Abs{P}$, give a combinatorial description of certain simplicial complexes.
However, not all complexes can be realised by this construction.
For example, consider $S^1$ realised as two edges connected at their ends.
This limitation is linked to the following observation.

All the information of a poset $P$ is encoded exactly by the following category $\catc$:
The objects of  $\catc$ are the elements of  $P$, and there is a morphism  $x \to y$ in  $\catc$ exactly when  $x \leq y$ in $P$.
The category $\catc$ is \emph{thin}, meaning that for all  $x,y \in \ob(\catc)$, there is at most one morphism  $x \to y$.
Bearing this in mind, with the aim of being able to construct $S^1$ in a combinatorial way, we give the following definition.

\begin{definition}
	A small category without loops (abbreviated scwol), is a small category $\catx$ such that for all composable morphisms $x_1 \to x_2 \to \cdots \to x_n$, if $x_1=x_n$, then $x_1=x_i$ for all $i$, and each morphism is $\id_{x_1}$.
	\label{def:scwol}
\end{definition}

Note that this definition means that no two distinct objects are isomorphic.
It also means that the only morphism $x \to x$ is $\id_x$.
Conversely, if a small category satisfies both of these properties, then it is a scwol.

With geometry in mind, we call the objects of a scwol $\catx$ \emph{vertices} and denote the set of vertices by $V(\catx)$.
We call the non--identity morphisms of  $\catx$ \emph{edges} and denote the set of edges by $E(\catx)$.
Given some $a \in E(\catx)$, we denote the initial (source) and terminal (target) vertices by $i(a)$ and $t(a)$ respectively.
If two edges $b$ and $a$ satisfy $t(b)=i(a)$, then we say that $a$ and $b$ are composable and denote their composition $ab$, which is necessarily another edge in $E(\catx)$.
We have that $i(ab) = i(b)$ and $t(ab) = t(a)$.
We now work to construct a geometric realisation $\Abs{\catx}$ of a scwol $\catx$, such that if $\catx$ is thin, then $\Abs{\catx}$ matches the geometric realisation when considering $\catx$ as a poset.

Let $E^{(n)}(\catx)$ denote the set of all $n$--tuples of composable edges in $\catx$,
i.e.~if $(a_1, a_2, \ldots, a_n) \in E^{(n)}(\catx)$, then
\[
	\begin{tikzcd}
		x_1 \ar[r, "a_1"] & x_2 \ar[r, "a_2"] & {} \ar[r, phantom, "\cdots"] & {} \ar[r, "a_n"] & x_{n+1}
	\end{tikzcd}
\]
is a composable set of edges in $\catx$.
By convention, $E^{(0)}(\catx) = V(\catx)$.
We define the following maps, $\partial_i \colon E^{(n)}(\catx) \to E^{(n-1)}(\catx)$ on composable tuples, where
\begin{align*}
	\partial_0(a_1,\ldots,a_n) & = (a_2,\ldots,a_n)                                          \\
	\partial_i(a_1,\ldots,a_n) & = (a_1, \ldots, a_ia_{i+1}, \ldots, a_n) \quad 1 \leq i < k \\
	\partial_n(a_1,\ldots,a_n) & = (a_1, \ldots, a_{n-1})
	.\end{align*}

And maps, $d_i \colon \Delta^{k-1} \to \Delta^k$ on simplices, where
\[
	d_i(t_0, \ldots, t_{k-1}) = (t_0, \ldots, t_{i-1}, 0, t_i, \ldots, t_{k-1}) \quad 0 \leq i < k
	.\]

\begin{definition}
	Given a scwol $\catx$, the geometric realisation, $\Abs{\catx}$ is a simplicial complex defined as
	\[
		\Abs{\catx} \coloneq \bigsqcup_k \left( \Delta^k \times E^{(k)}(\catx) \right) / \sim
		.\]
	Where $\sim$ is the transitive closure of the relations
	\[
		(d_i(x),t) \sim (x, \partial_i(t))
	\]
	across all tuples $t \in E^{(k)}(\catx)$.
\end{definition}
In this definition, we had to take more care than with the geometric realisation of a poset.
This is because, in a poset, a composable tuple is completely determined by its vertices (the chain), so the combinatorial face maps always corresponded to taking subsets of those vertices.
But in a scwol, we need to account for potentially multiple edges between two vertices, so we must enumerate simplices using edges, rather than vertices.

For those interested, the above construction of $\Abs{\catx}$ corresponds to the standard geometric realisation of the nerve (which corresponds to the union of all of our $E^{(k)}(\catx)$) of $\catx$.
See \cite{goerss_simplicial_2009}.

\begin{example}
	The following scwol has geometric realisation $S^1$, constructed by identifying the ends of two 1--simplices.
	\begin{center}
		\begin{tikzpicture}
			\tikzstyle{every label}=[font=\footnotesize]
			\tikzstyle{every node}=[font=\footnotesize]
			\node[FSC] (base)	at (0,0)	{};
			\node[FSC] (top)	at (0,1)	{};
			\draw[arrow_me=stealth, bend left=90] (base) to (top);
			\draw[arrow_me=stealth, bend right=90] (base) to (top);
		\end{tikzpicture}
	\end{center}
\end{example}

\begin{example}
	Any polygon can be realised as the geometric realisation of a scwol $\catx$ in the following way:
	Consider the polygon as a polygonal complex.
	Make a vertex for at the centre of each polygon in the complex.
	There is an edge from centres of polygons to centres of faces of that polygon.
	The below picture shows the scwol from this construction applied to a square.
	\begin{center}
		\begin{tikzpicture}
			\def\sqsize{2}
			\tikzstyle{every label}=[font=\footnotesize]
			\tikzstyle{every node}=[font=\footnotesize]
			\node[FSC] (bl)	at (0,0)	{};
			\node[FSC] (br)	at ($(bl)+(\sqsize,0)$)	{};
			\node[FSC] (tl)	at ($(bl)+(0,\sqsize)$)	{};
			\node[FSC] (tr)	at ($(tl)+(\sqsize,0)$)	{};

			\node[FSC] (mblbr) at ($(bl)!0.5!(br)$) {};
			\node[FSC] (mbltl) at ($(bl)!0.5!(tl)$) {};
			\node[FSC] (mbrtr) at ($(br)!0.5!(tr)$) {};
			\node[FSC] (mtltr) at ($(tl)!0.5!(tr)$) {};

			\node[FSC] (center) at ($(bl)!0.5!(tr)$) {};

			\draw[arrow_me=stealth] (center) to (bl);
			\draw[arrow_me=stealth] (center) to (br);
			\draw[arrow_me=stealth] (center) to (tl);
			\draw[arrow_me=stealth] (center) to (tr);

			\draw[arrow_me=stealth] (center) to (mblbr);
			\draw[arrow_me=stealth] (center) to (mbltl);
			\draw[arrow_me=stealth] (center) to (mtltr);
			\draw[arrow_me=stealth] (center) to (mbrtr);

			\draw[arrow_me=stealth] (mblbr) to (bl);
			\draw[arrow_me=stealth] (mblbr) to (br);
			\draw[arrow_me=stealth] (mbltl) to (bl);
			\draw[arrow_me=stealth] (mbltl) to (tl);
			\draw[arrow_me=stealth] (mbrtr) to (br);
			\draw[arrow_me=stealth] (mbrtr) to (tr);
			\draw[arrow_me=stealth] (mtltr) to (tl);
			\draw[arrow_me=stealth] (mtltr) to (tr);
		\end{tikzpicture}
	\end{center}
\end{example}

\begin{example}
	Following the same procedure as the previous example, the scwol associated to the complete graph on three vertices is given below.
	\begin{center}
		\begin{tikzpicture}
			\def\trisize{1.5}
			\tikzstyle{every label}=[font=\footnotesize]
			\tikzstyle{every node}=[font=\footnotesize]

			\node[FSC] (bl)	at (0,0)	{};
			\node[FSC] (br)	at ($(bl)+(\trisize,0)$)	{};
			\node[FSC] (top) at ($(bl)+(0.5*\trisize,0.833*\trisize)$)	{};

			\node[FSC] (mblbr) at ($(bl)!0.5!(br)$) {};
			\node[FSC] (mblt) at ($(bl)!0.5!(top)$) {};
			\node[FSC] (mbrt) at ($(br)!0.5!(top)$) {};

			\draw[arrow_me=stealth] (mblbr) to (bl);
			\draw[arrow_me=stealth] (mblbr) to (br);
			\draw[arrow_me=stealth] (mblt) to (bl);
			\draw[arrow_me=stealth] (mblt) to (top);
			\draw[arrow_me=stealth] (mbrt) to (br);
			\draw[arrow_me=stealth] (mbrt) to (top);
		\end{tikzpicture}
	\end{center}
\end{example}





