\todo[color=olive]{Hello Sean, it's Ala.
	Feel free to ignore any suggestions I make - I chose a different colour so its clear which are my notes.
	I also corrected any grammar and punctuation that I noticed, but I might have not caught everything.}
In this section we will introduce the construction of \emph{complexes of groups}, largely due to Haefliger \cite{haefliger_complexes_1991}.
This generalises the construction of graphs of groups due to Bass and Serre, discussed in \cref{subsec:graphs_of_groups}.
A related construction of \emph{triangles of groups} was studied by Gersten and Stallings \cite{stallings_nonpositively_1991} and complexes of groups were studied in two dimensions, independently of Haefliger, by Corson \cite{corson_complexes_1992}.
The exposition here largely follows the beginning of \cite[Chapter 3.\textrm{\ensuremath{\calc}}]{BrHa11}, with some modification to notation and with some elements explored in more detail.
Specifically, all the examples and the exposition which motivates the details of \cref{def:scwol,def:complex_of_groups,def:complex_of_groups_from_action,def:morhpism_of_complexes_of_groups,def:paths_in_complexes_of_groups} is my own work.

The construction of graphs of groups arose by abstracting emergent features of group actions on trees.
Analogously, to define complexes of groups, we will consider certain group actions on geometric complexes associated to so-called \emph{small categories without loops}.
However, unlike with graphs of groups, not every complex of groups can be realised as a quotient of group action on a complex.
