\subsection{The classifying space of a complex of groups}
We have already discussed how to create spaces from scwols.
In fact, we may view scwols as a discrete model of such spaces.
We now work to define a space related to a complex of groups.
The fundamental group of this space can be computed directly from the algebraic structure of the complex of groups.

Given a scwol $\calx$, homotopy in $\abs{\calx}$ is well modelled by the category structure in $\calx$.
Let  $E^{\pm}(\calx)$ denote abstract symbols, that should be thought of as \emph{oriented edge paths} in $\calx$.
We set $i(a^+) = i(a)$, $t(a^+) = t(a)$, $i(a^-) = t(a)$ and $t(a^-)= i(a)$.
For example, we may think of $a^-$ as the isometric embedding $\gamma \colon [0,1] \to a \subseteq \abs{\calx}$ such that  $\gamma(0) = t(a)$ and $\gamma(1) = i(a)$.
We should also think of $(\_)^\pm$ as a map $E^\pm(\calx) \to E^\pm(\calx)$ in the obvious way, where  $(a^-)^-= a^+$ etc.
An edge path from $\sigma$ to  $\tau$ in $\abs{\calx}$ is a tuple $(e_1, \ldots, e_n) \in \left(E^\pm(\calx)\right)^n$ such that $i(e_1)=\sigma$,  $t(e_n)=\tau$ and  $t(e_i) = i(e_{i+1})$ for all  $1 \leq i < n$.
We can define homotopy of such edge paths in the following way.
Given an edge path $(e_1, \ldots, e_n) \in \left(E^\pm(\calx)\right)^n$, we may do the following replacements:
\begin{enumerate}
	\item Any adjacent subpath $(e_i,e_{i+1})$, where $e_i=e_{i+1}^-$ can be deleted.
	\item Any adjacent subpath $(e_i,e_{i+1})$, where $e_i = b^+$ and  $e_{i+1}=a^+$ can be replaced by the subpath $((ab)^+)$, which is guaranteed to exist.
	      Similarly, if $e_i = b^-$ and  $e_{i+1}=a^-$, we can replace $(e_i,e_{i+1})$ with  $((ba)^-)$.
\end{enumerate}
A homotopy of edge paths is a sequence of such edge path replacements, and their inverses.
We can then define the fundamental group of $\calx$,  $\pi_1(\calx,\sigma_0)$, to be the homotopy class of edge paths that start and end at  $\sigma_0$.
This is isomorphic to the usual $\pi_1(\abs{\calx}, \sigma_0)$ by \cite[Corollary 4.12]{hatcher_algebraic_2001}.

We now define a category whose topology (via the geometric realisation) encodes some of the algebra of a complex of groups.
\begin{definition}
	Given a complex of groups $\calg$ over the scwol  $\calx$, we define the category $C\calg$ as follows.
	The objects of  $C\calg$ are the objects (vertices) of $\calx$.
	The morphisms of $C\calg$ are the tuples $(g,\alpha) \colon i(\alpha) \to t(\alpha)$, where  $g \in G_{t(\alpha)}$ and  $\alpha$ is a (potentially identity) morphism in $\calx$.
	The composition  $(g,\alpha)(h,\beta)$ exists when  $t(\beta) = i(\alpha)$ in  $\calx$, and it is defined to be
	\[
		(g,\alpha)(h,\beta) \coloneq (g\phi_\alpha(h)g_{\alpha,\beta}, \alpha\beta)
		.\]
	For this definition, we take $\phi_{\alpha}$ to be $\id_{G_\sigma}$ if $\alpha = \id_\sigma$, and $g_{\alpha,\beta}$ to be the identity in $G_{t(\alpha)}$ if either of $\alpha$ or  $\beta$ are identity morphisms.
\end{definition}
Compatibility condition (2) in \cref{def:complex_of_groups} guarantees associativity of this composition in  $C\calg$.
If we take $\calg$ to be the complex of groups of a group  $G$ over the trivial scwol, then the 2-skeleton of $\abs{C\calg}$  is the presentation complex of $G$ with generating set $G$.
\begin{definition}
	Given a complex of groups $\calg = (G_\sigma, \phi_a, g_{a,b})$ over the scwol $\calx$, we define the \emph{universal group} associated to $\calg$ to be
	\[
		F\calg \coloneq  \GroupPres{E^\pm(\calx) \sqcup \bigsqcup_{\sigma \in V(\calx)} G_\sigma \relations R}
		.\]
	Where $R$ is all relations of the following form.
	\[
		R \coloneq \left\{
		\begin{array}{l}
			\text{All identity elements are equal} \\
			\text{All relations in each }G_\sigma  \\
			(a^+)^{-1}   = a^-                     \\
			(a^-)^{-1}   = a^+                     \\
			a^+b^+(ab)^- = g_{a,b}                 \\
			\phi_a(g)    = a^+ga^-
		\end{array}
		\right\}
	\]
	for all $a,b,ab \in E(\calx)$ and $g \in G_\sigma$ for which these expressions are well-defined.
\end{definition}
There is a natural morphism $i \colon \calg \to F\calg$ where $g \in G_\sigma$ is mapped to the corresponding generator in $F\calg$, and  $a \in E(\calx)$ is sent to $a^+$.
The relations in $F\calg$ are exactly the ones needed to make  $i$ a morphism in the sense of \cref{def:morphism_of_complex_of_groups_to_group}.
In this context, the $\theta_a$ correspond to the $a^+$ generators in  $F\calg$.
