\newpage % I will remove this at the end (just find it easier to focus this way while it's a work-in-progress)
\section{Talia's section}

% Keep in mind Matt's recent feedback: (1) try to make links to wider topics in GGT to allow readers other avenues of interest, (2) clearly (and proudly!) state when a result is your own

\subsection{Introduction}

% I'm not sure on the wording of this, but for now I'll get some ideas down and refine them later
We finally reach the topic of ends of groups. In this chapter, we explore the relationship between the number of ends of a finitely generated group and the algebraic structure of these groups as amalgams and HNN-extensions. An important result in this area is Stallings’ Structure Theorem, which classifies finitely generated groups with more than one end:

\begin{theorem}[Stallings' Structure Theorem]
A finitely generated group with more than one end is either:
    \begin{enumerate}
        \item virtually infinite cyclic, or
        \item splits as a non-trivial free product over a finite subgroup, or
        \item can be written as an HNN-extension over a finite subgroup.
    \end{enumerate}
\end{theorem}


Here we follow the method of Kr{\"o}n's paper \cite{K10} to prove Stallings' Theorem using a construction of \emph{structure cuts}. Originally introduced by Dunwoody in \cite{D79}, these are automorphism invariant tree-decompositions of graphs based on the principle of removing finitely many edges. % Put this last bit in my own words
We then introduce the cohomological definition of an end of a group, and discuss related results in this context. % Edit this once finished. Mention accessibility?

\subsection{Structure trees}
 

\begin{definition}
    
\end{definition}

\begin{proposition}
Let \(G\) be an infinite finitely generated group with generating set \(S\).
The group \(G\) has more than one end if and only if there is a subset \(C\) of \(G\) such that \(C\) and \(G \setminus C\) are infinite and \(Cg \setminus C\) is finite, for all \(g \in G\).
\end{proposition}

\begin{proof}
    
\end{proof}
\newpage % I will remove this at the end