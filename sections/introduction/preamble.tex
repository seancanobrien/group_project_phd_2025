\todo{hi all, I have started to write some stuff.
	Feel free to edit this.
S}

Given a group $G$, generated by a subset $S \subseteq G$, we may construct a space on which the group acts freely and transitively, namely its Cayley graph.
In this sense, all groups are (a subgroup of) a symmetry group of a space.
Fixing a group $G$, we may have many such spaces $X$ and actions $G \acts X$, which we may use to reason about the group $G$.
Broadly, geometric group theory involves exploring spaces that encode information about groups in this way.
If we restrict our study to finitely presented groups, the spaces that often turn out to be useful are those that can be defined in a discrete, procedural way, such as CW complexes, simplicial complexes, metric spaces, or graphs.
\todo[color=olive]{I think that these paragraphs don't really fit together right now - I tried to make an overview subsection, but that caused a weird white space to appear...}

\textcolor{olive}{Would we want to talk more about the history of GGT or the specific things we are considering?}

\textcolor{pink}{Something like this? -Lorna} \textcolor{olive}{Yes, that looks great}
This paper is mainly concerned with group actions on trees, in particular the theory developed by Bass and Serre in the 1970's. The theory gives a way to study these groups by building a \emph{graph of groups} that represents certain elements of group structure. We explore some of these elements in more detail, as well as some ways in which the theory can be generalised.  

\subsection{Overview} \textcolor{olive}{Do we want a subsection, or just paragraphs?}
We begin by exploring \emph{Baumslag-Solitar groups} and their connection to \emph{Bass-Serre theory}, which concerns actions of groups on trees and their relationship to certain graphs of groups.
Baumslag-Solitar groups were first introduced in \cite{BajoHNN} to provide examples of one-relator \emph{non-Hopfian} groups. Thus, we start the section by considering the property of being Hopfian, and showing that $BS(2,3)$ indeed is not. 

Other topics that we explore early in the section are the answer to the \emph{isomorphism problem}, \emph{residual finiteness} and condition for \emph{solvability} of $BS(m,n)$. Afterwards we proceed to focus on the fact that Baumslag-Solitar groups can be viewed as a specific type of extensions of $\Z$, namely \emph{HNN extensions} (named after G. Higman, B. Neumann and H. Neumann). That point of view gives us a way to consider these groups through the lens of Bass-Serre theory, which is developed side by side with applying it to our specific example. We define both \emph{graphs of groups} and \emph{$G$-trees} in the sense of Serre \cite{serre_trees_1980}, and exhibit Baumslag-Solitar groups as fundamental groups of specific graphs consisting of one loop. Finally, we construct the Bass-Serre tree for $BS(m,n)$.

After exploring properties of classical Baumslag-Solitar groups we state their generalisation - the \emph{generalised Baumslag-Solitar ($GBS$)} groups. We start by considering some examples and introduce the concept of an \emph{elementary $GBS$} groups. We follow the work of Forester \cite{For03}, Levitt \cite{Le15} and Whyte \cite{WH01} while we explore properties of $GBS$ graphs, trees and groups that come with them. One of the key results we discuss is the classification of $GBS$ graphs due to Whyte, which also tells us information about quasi-isometies between some of $BS(m,n)$ groups. Finally, we look at quotients and subgroups that occur for $GBS$ groups.

We finish the section on Baumslag-Solitar groups and their generalisation by referring the reader to other interesting questions and work on the subject. Thus, one can treat this part of the paper as a survey of how different properties weave together for one specific class of groups.

In \cref{sec:complexes_of_groups}, we explore \emph{complexes of groups}.
These should be understood as higher-dimensional analogues of graphs of groups.
With graphs of groups, we assign groups to the vertices and edges of a graph.
We then assign two injective homomorphisms from each edge group in to the vertex groups at each of the edge's ends.
In this way, none of the homomorphisms in a graph of groups are composable, as shown in the following diagram.
% https://q.uiver.app/#q=WzAsNyxbMiwwLCJHX3t2XzB9Il0sWzAsMCwiR197dl8xfSJdLFs0LDAsIkdfe3ZfMn0iXSxbMiwyLCJHX3t2XzN9Il0sWzIsMSwiR197ZV8zfSJdLFsxLDAsIkdfe2VfMX0iXSxbMywwLCJHX3tlXzJ9Il0sWzQsMCwiIiwwLHsic3R5bGUiOnsidGFpbCI6eyJuYW1lIjoiaG9vayIsInNpZGUiOiJ0b3AifX19XSxbNCwzLCIiLDIseyJzdHlsZSI6eyJ0YWlsIjp7Im5hbWUiOiJob29rIiwic2lkZSI6ImJvdHRvbSJ9fX1dLFs1LDAsIiIsMix7InN0eWxlIjp7InRhaWwiOnsibmFtZSI6Imhvb2siLCJzaWRlIjoidG9wIn19fV0sWzUsMSwiIiwwLHsic3R5bGUiOnsidGFpbCI6eyJuYW1lIjoiaG9vayIsInNpZGUiOiJib3R0b20ifX19XSxbNiwyLCIiLDAseyJzdHlsZSI6eyJ0YWlsIjp7Im5hbWUiOiJob29rIiwic2lkZSI6InRvcCJ9fX1dLFs2LDAsIiIsMCx7InN0eWxlIjp7InRhaWwiOnsibmFtZSI6Imhvb2siLCJzaWRlIjoiYm90dG9tIn19fV1d
\[\begin{tikzcd}
		{G_{v_1}} & {G_{e_1}} & {G_{v_0}} & {G_{e_2}} & {G_{v_2}} \\
		&& {G_{e_3}} \\
		&& {G_{v_3}}
		\arrow[hook', from=1-2, to=1-1]
		\arrow[hook, from=1-2, to=1-3]
		\arrow[hook', from=1-4, to=1-3]
		\arrow[hook, from=1-4, to=1-5]
		\arrow[hook, from=2-3, to=1-3]
		\arrow[hook', from=2-3, to=3-3]
	\end{tikzcd}\]

Complexes of groups are higher-dimensional in an algebraic sense, because they explore systems of groups and injective homomorphisms in which the homomorphisms are composable.
The structure of a category keeps track of these map compositions, and the categories that are useful to complexes of groups are so-called \emph{small complexes without loops}, abbreviated as scwols.

We will introduce some constructions from category theory to motivate the definition of scwols.
A poset $P$ can be modelled by a category $\calc$ where the objects of $\C$ are the elements of $P$, and there is a unique morphism $p \to q$ exactly when $p \leq q$.
In such a category, we never have a chain $p < q < p$, which would correspond to a directed loop in the corresponding category.
Scwols are similar, in that they are \emph{without loops}, but do not have the restriction that two distinct elements have at most one morphism between them.
It is very reasonable to have different injective homomorphisms from one group to another, but we should not consider the case where there is a cycle of such homomorphisms, which would necessitate all the injective homomorphisms to be isomorphisms.
In this way, scwols model this situation well

Being a category, a scwol $\calx$ can also model a space, namely $\abs{\calx}$, the geometric realisation of its nerve.
This space $\abs{\calx}$ is a simplicial complex, where $n$--simplices correspond to tuples of $n$ composable morphisms in $\calx$.
The face maps are given by morphism composition in $\calx$.

A complex of groups is an assignment of a group to each object in a scwol, and an injective homomorphism to each morphism in the scwol, with some additional data that encodes composition.
As such, for each composable pair of morphisms
\[
	\begin{tikzcd}
		\sigma \ar[r, "b"] & \tau \ar[r, "a"] & \nu
	\end{tikzcd}
\]
in $\calx$, there are three injective homomorphisms in the corresponding complex of groups.
Namely, $\phi_b \colon G_\sigma \to G_\tau$, $\phi_a \colon G_\tau \to G_\nu$, and $\phi_{ab} \colon G_\sigma \to G_\nu$.
We do not require that composition be exactly $\phi_a\phi_b = \phi_{ab}$, as we would expect in a category, but allow for the composition to be off by conjugation by some element $g_{a,b} \in G_\nu$, which we keep track of, i.e.~$\ad(g_{a,b})\phi_{ab} = \phi_a\phi_b$.

Complexes of groups may arise from group actions on scwols, and the construction of the complex of groups associated to the quotient scwol is very similar to the construction with graphs of groups.
Accordingly, the local groups are associated to stabilisers of the action.
We call any graph of groups that arise from such an action \emph{developable}.

Complexes of groups also emerge from geometric actions.
Suppose we have some group action $G \acts Y$, where $Y$ is some polyhedral complex.
We may model this polyhedral complex with a scwol $\calx$, such that $\abs{\calx}$ is naturally homeomorphic to the barycentric subdivision of $Y$.
We then have an action $G \acts \abs{\calx}$ and $G \acts \calx$.
From this, we can get a complex of groups over the quotient scwol associated to $G \acts \calx$.
So we can use complexes of groups to encode group actions of polyhedral complexes.

We conclude the section on complexes of groups by giving the construction of the fundamental group of a complex of groups.
When the complex of groups arose from the action of a group $G$ on a scwol, the fundamental group is a way of recovering the original group $G$.
Graphs of groups always have an associated action on a tree, but this is not true with complexes of groups.
We see the developability of a complex of groups depends on whether the natural homomorphisms from the local groups to the fundamental group are injections.

In Section 4, we delve into the topic of ends of groups. Intuitively, ends of groups describe the connected components of a topological space at infinity. For a simple example, consider the real line. Intuitively, we can see this space has two `ends', similarly the (real) plane has one end. A natural start to this section is therefore to formalise a definition of ends. In this section, we focus specifically on ends of finitely generated groups, where ends relate to the components at infinity of a Cayley graph for the group up to a choice of finite generating set. 

The main focus of this section is Stallings' Structure Theorem, which classifies finitely generated groups with more than one end as either (i) virtually cyclic groups or (ii) groups which are \emph{splittings} over a finite subgroup. These splittings are defined as \emph{amalgams} and \emph{HNN extensions} --- the latter of which is introduced in Section 2. Splittings offer a different perspective on these extensions: instead of adding structure on top of our base group, amalgams and HNN extensions allow us to divide or `factorise' these groups in a certain way. 

After introducing ends, their basic properties and some algebraic background on splittings of groups, we progress toward showing a way to construct splittings of multiple ended groups. To do this, we follow Kr\"{o}n's method in \cite{K10}, which introduces a notion of \emph{cuts}. These are sets of vertices in a graph which contain the vertices of an infinite, non-backtracking path without loops, as well as some additional conditions. We show various properties of cuts before demonstrating an example in \(\mathrm{PSL}(2,\mathbb{Z})\). Finally, we outline a sketch of the proof of Stallings' structure theorem using cuts and a related construction of \emph{structure trees}. We also highlight several areas for further research, namely additional examples of one-ended groups and a generalised notion of ends.

The classification given by Stallings' theorem is interesting on several levels. Firstly, it is surprising that having some information on the behaviour of a group at infinity (i.e. its number of ends) can tell us about the algebraic structure of the group as a whole. In addition, it tells us that groups with more than one end are quite sparse, as they must fall into these (relatively) small classes. Furthermore, in the context of Bass--Serre theory, splittings of a group can be seen as an action on a tree. Specifically, a splitting of a group \(G\) over a subgroup \(H\) can equivalently be defined as a transitive and non-trivial action of \(G\) on a tree with \(H\) an edge stabiliser. Therefore, Stallings' theorem can also be understood in terms of actions of groups on trees; linking back to our main theme.

\todo[color=green!40]{I think this feels like a bit of a jump - the transition from Section 4 to 5 could be smoother. Talia}
Finally, we discuss actions of groups on metric spaces which are a slight generalisation of trees. \todo[color=pink]{I've addes a sentence here}One way to define a (\emph{simplicial}) tree is as a graph in which there is precisely one path between any two points. Section 5 explores the consequences of relaxing the word `graph' in this definition to `metric space' in general. This gives us the definition of $\mathbb{R}$-trees. As we see in the first three sections, group actions on trees are relatively well-understood through Bass--Serre theory. However, problems are encountered in attempts to apply these techniques to $\mathbb{R}$-trees. For example, even the Fundamental Theorem of Bass--Serre Theory, which allows one to recover a group from a tree on which it acts, does not hold in this slightly more general class of metric spaces. However, $\mathbb{R}$-trees and the groups which act on them have played a significant role in geometric group theory, appearing in the study of automorphisms of several different classes of groups. The final section of the paper aims to highlight some of the ways in which $\mathbb{R}$-trees and simplicial trees differ, as well as giving an introduction to the alternative methods used in the study of $\mathbb{R}$-trees. Finally, we describe some of the applications of these metric spaces, aiming to give the reader an idea of their value.

After defining $\mathbb{R}$-trees, we give some examples, including examples of $\mathbb{R}$-trees which are not simplicial. One way in which $\mathbb{R}$-trees arise is as limits of hyperbolic metric spaces (under a suitable notion of convergence), and we explore this construction and some of its consequences. 

We then move on to group actions on $\mathbb{R}$-trees, beginning by classifying the isometries of these spaces. As in many areas of geometric group theory, there are several notions of `nice' group actions; the ones of interest here are free, \emph{non-trivial}, and \emph{stable} actions. 

The majority of the section covers \emph{band complexes}; a band complex is a certain type of relative CW complex which describes the action of a group $G$ on an $\mathbb{R}$-tree. This brings us to the \emph{Rips Machine}, an algorithm which takes a band complex and transforms it into a `normal form': a disjoint union of smaller subcomplexes, each of which is of one of four types. These types are \emph{simplicial}, \emph{surface}, \emph{toral}, and \emph{thin}, and they describe features of the structure of $G$ in a similar way to that in which graphs of groups describe the structure of groups acting on trees.

Finally, we give a brief survey of the applications of $\mathbb{R}$-trees. Their appearance as limits of hyperbolic spaces mean that they arise particularly often in the study of hyperbolic groups, and we state some of these results here. Also discussed is Marc Culler and Karen Vogtmann's \emph{Outer Space}, a powerful tool in the study of automorphisms of free groups and another utilisation of $\mathbb{R}$-trees. 
\todo[color=olive]{How to avoid this white space? ((It is because of the \\newpage before Alicja's sec. We will remove this before final compile, so is not a problem. S))}

\subsection{Preliminary definitions}
\todo[color=olive]{For now I will just list the definitions that we have in common -  if we decide to not proceed with this section, I will delete it}
Common definitions:
\begin{enumerate}
    \item Graph $\Gamma$ - we are not using the same notion of a graph everywhere( I believe Sean and myself use graphs in the sense of Serre, but I am not sure about other sections) \textcolor{pink}{I'm saying `simplicial' or `metric' graph in my section to distinguish between these and Serre's. Not sure about Talia } \textcolor{cyan}{Talia here, I think mine agree with Serre's except that all my graphs are undirected} \textcolor{olive}{Maybe we should then not define graphs and specify them in our sections? Otherwise if we do define graphs, we should define all the ones that appear, not only the definition in Serre}
    \item HNN extension, Amalgamated product, splitting \textcolor{pink}{If we're not doing the other two maybe we should leave this one too? Might be a bit odd to just have one or two definitions here}
    \item basic category theory definitions were proposed - I am not sure if they are used anywhere else apart from Section on complexes of groups, but we could put them here \textcolor{olive}{Probably not doing that. (Yep, I think I have decided against putting any cat theory in the intro. S.)}
\end{enumerate}
