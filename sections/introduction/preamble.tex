\todo{hi all, I have started to write some stuff.
Feel free to edit this. S}

Given a group $G$, generated by a subset $S \subseteq G$, we may construct a space on which the group acts freely and transitively, namely its Cayley graph.
In this sense, all groups are (a subgroup of) a symmetry group of a space.
Fixing a group $G$, we may have many such spaces $X$ and actions $G \acts X$, which we may use to reason about the group $G$.
Broadly, geometric group theory involves exploring spaces that encode information about groups in this way.
If we restrict our study to finitely presented groups, the spaces that often turn out to be useful are those that can be defined in a discrete, procedural way, such as CW complexes, simplicial complexes, or graphs.

This paper begins by exploring Baumslag-Solitar groups. These were first introduce to provide interesting examples of \emph{non-Hopfian} groups. and some of their algebraic properties are explored...
\textcolor{red}{Mention what things were done in order}
\todo{Alicja can do a better job of this I think :) -S}
We introduce Bass-Serre theory and graphs of groups....

In \cref{sec:complexes_of_groups}, we explore \emph{complexes of groups}. These should be understood as higher-dimensional analogues of graphs of groups.
With graphs of groups, we assign groups to the vertices and edges of a graph.
We then assign an two injective homomorphisms from each edge group in to the vertex groups at each of the edge's ends.
In this way, none of the homomorphisms in a graph of groups are composable, as shown in the following diagram.
% https://q.uiver.app/#q=WzAsNyxbMiwwLCJHX3t2XzB9Il0sWzAsMCwiR197dl8xfSJdLFs0LDAsIkdfe3ZfMn0iXSxbMiwyLCJHX3t2XzN9Il0sWzIsMSwiR197ZV8zfSJdLFsxLDAsIkdfe2VfMX0iXSxbMywwLCJHX3tlXzJ9Il0sWzQsMCwiIiwwLHsic3R5bGUiOnsidGFpbCI6eyJuYW1lIjoiaG9vayIsInNpZGUiOiJ0b3AifX19XSxbNCwzLCIiLDIseyJzdHlsZSI6eyJ0YWlsIjp7Im5hbWUiOiJob29rIiwic2lkZSI6ImJvdHRvbSJ9fX1dLFs1LDAsIiIsMix7InN0eWxlIjp7InRhaWwiOnsibmFtZSI6Imhvb2siLCJzaWRlIjoidG9wIn19fV0sWzUsMSwiIiwwLHsic3R5bGUiOnsidGFpbCI6eyJuYW1lIjoiaG9vayIsInNpZGUiOiJib3R0b20ifX19XSxbNiwyLCIiLDAseyJzdHlsZSI6eyJ0YWlsIjp7Im5hbWUiOiJob29rIiwic2lkZSI6InRvcCJ9fX1dLFs2LDAsIiIsMCx7InN0eWxlIjp7InRhaWwiOnsibmFtZSI6Imhvb2siLCJzaWRlIjoiYm90dG9tIn19fV1d
\[\begin{tikzcd}
	{G_{v_1}} & {G_{e_1}} & {G_{v_0}} & {G_{e_2}} & {G_{v_2}} \\
	&& {G_{e_3}} \\
	&& {G_{v_3}}
	\arrow[hook', from=1-2, to=1-1]
	\arrow[hook, from=1-2, to=1-3]
	\arrow[hook', from=1-4, to=1-3]
	\arrow[hook, from=1-4, to=1-5]
	\arrow[hook, from=2-3, to=1-3]
	\arrow[hook', from=2-3, to=3-3]
\end{tikzcd}\]

Complexes of groups are higher-dimensional in an algebraic sense, because they explore systems of groups and injective homomorphisms in which the homomorphisms are composable.
The structure of a category keeps track of these map compositions, and the categories that are useful to complexes of groups are so-called \emph{small complexes without loops}, abbreviated as scwols.

We will introduce some constructions from category theory to motivate the definition of scwols.
A poset $P$ can be modelled by a category $\calc$ where the objects of $\C$ are the elements of $P$, and there is a unique morphism $p \to q$ exactly when $p \leq q$.
In such a category, we never have a chain $p < q < p$, which would correspond to a directed loop in the corresponding category.
Scwols are similar, in that they are \emph{without loops}, but do not have the restriction that two distinct elements have at most one morphism between them.
It is very reasonable to have different injective homomorphisms from one group to another, but we should not consider the case where there is a cycle of such homomorphisms, which would necessitate all the injective homomorphisms to be isomorphisms.
In this way, scwols model this situation well

Being a category, a scwol $\calx$ can also model a space, namely $\abs{\calx}$, the geometric realisation of its nerve.
This space $\abs{\calx}$ is a simplicial complex, where $n$--simplices correspond to tuples of $n$ composable morphisms in $\calx$.
The face maps are given by morphism composition in $\calx$.

A complex of groups is an assignment of a group to each object in a scwol, and an injective homomorphism to each morphism in the scwol, with some additional data that encodes composition.
As such, for each composable pair of morphisms  
\[
\begin{tikzcd}
    \sigma \ar[r, "b"] & \tau \ar[r, "a"] & \nu
\end{tikzcd}
\] 
in $\calx$, there are three injective homomorphisms in the corresponding complex of groups. Namely $\phi_b \colon G_\sigma \to G_\tau$, $\phi_a \colon G_\tau \to G_\nu$, and $\phi_{ab} \colon G_\sigma \to G_\nu$.
We do not require that composition be exactly $\phi_a\phi_b = \phi_{ab}$, as we would expect in a category, but allow for the composition to be off by conjugation by some element $g_{a,b} \in G_\nu$, which we keep track of, i.e.~$\ad(g_{a,b})\phi_{ab} = \phi_a\phi_b$.

Complexes of groups may arise from group actions on scwols, and the construction of the complex of groups associated to the quotient scwol is very similar to the construction with graphs of groups.
Accordingly, the local groups are associated to stabilisers of the action.
We call any graph of groups that arise from such an action \emph{developable}.

Complexes of groups also emerge from geometric actions.
Suppose we have some group action $G \acts Y$, where $Y$ is some polyhedral complex.
We may model this polyhedral complex with a scwol $\calx$, such that $\abs{\calx}$ is naturally homeomorphic to the barycentric subdivision of $Y$.
We then have an action $G \acts \abs{\calx}$ and $G \acts \calx$.
From this, we can get a complex of groups over the quotient scwol associated to $G \acts \calx$.
So we can use complexes of groups to encode group actions of polyhedral complexes.

We conclude the section on complexes of groups by giving the construction of the fundamental group of a complex of groups.
When the complex of groups arose from the action of a group $G$ on a scwol, the fundamental group is a way of recovering the original group $G$.
Graphs of groups always always have an associated action on a tree, but this is not true with complexes of groups. We see the developability of a complex of groups depends on whether the natural homomorphisms from the local groups to the fundamental group are injections.
